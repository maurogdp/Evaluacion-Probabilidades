\documentclass[11pt, letterpaper]{book}


\usepackage{pgfplots}
\pgfplotsset{compat=1.18}
\usepackage[utf8]{inputenc}
\usepackage[spanish]{babel}
\usepackage{amsmath, amsthm}
\usepackage{amsfonts}
\usepackage{amssymb}
\usepackage{graphicx}
\usepackage[left=2cm,right=2cm,top=2cm,bottom=2cm]{geometry}
\usepackage[export]{adjustbox}
\usepackage{multirow}
\usepackage{multicol}
\usepackage{setspace}
\usepackage{subfig}
\usepackage{venndiagram}
\usepackage{verbatim}
\usepackage{enumitem}
\usepackage{mdframed}
\usepackage{slashbox}


\usepackage{tikz}
\usetikzlibrary{arrows.meta,bbox}
\usetikzlibrary{decorations.pathreplacing}
\begin{comment}
\tikzset{%
  show curve controls/.style={
    postaction={
      decoration={
        show path construction,
        curveto code={
          \draw [blue] 
            (\tikzinputsegmentfirst) -- (\tikzinputsegmentsupporta)
            (\tikzinputsegmentlast) -- (\tikzinputsegmentsupportb);
          \fill [red, opacity=0.5] 
            (\tikzinputsegmentsupporta) circle [radius=.5ex]
            (\tikzinputsegmentsupportb) circle [radius=.5ex];
        }
      },
      decorate
}}}
\end{comment}
\usepackage{stackengine}
\newcommand\xrowht[2][0]{\addstackgap[.5\dimexpr#2\relax]{\vphantom{#1}}}


\usepackage{graphicx}
\usepackage{tikz}
\usetikzlibrary{babel,arrows.meta,decorations.pathmorphing, backgrounds,positioning,fit,petri, shapes, shadows}

\usepackage{tikz,color}
\usepackage{pgf-pie}
\usepackage{pgfplots} 

\theoremstyle{plain}% default
\newtheorem{teo}{Teorema}[section]
\newtheorem{lem}[teo]{Lema}
\newtheorem{prop}[teo]{Proposición}
\newtheorem*{cor}{Corolario}
\newmdtheoremenv{defi}{Definición}[section]
\newtheorem{conj}{Conjetura}[section]
\newtheorem{propi}{Propiedad}[section]

\theoremstyle{definition}
\newtheorem{ejer}{\textit{Ejercicio}}[section]
\newtheorem{ejem}{\textit{Ejemplo}}[section]
\newtheorem*{sol}{Solución}

\theoremstyle{remark}
\newtheorem{obs}{Observación}[section]
\newtheorem*{nota}{Nota}
\newtheorem{caso}{Caso}
\newtheorem*{tips}{Tips}




\newcommand{\paso}[1]{\hspace{1cm}\linebreak\hspace{1cm}\textit{#1 paso. }}


%%%%%%%%%%%%%%%%%%%%%%%%%%%%%%%%%%%%
%%%%%%%%%%%%%%%%%%%%%%%%%%%%%%%%%%%%
%%%%%%%%%%%%%%%%%%%%%%%%%%%%%%%%%%%%

\makeatletter
\newenvironment{myminipage}[1]%
    {\let\@parboxrestore\relax\begin{minipage}{#1}}%
    {\end{minipage}}
\makeatother
%%%%%%%%%%%%%%%%%%%%%%%%%%%%%%%%%%%%%%
%%%%%%%%%%%%%%%%%%%%%%%%%%%%%%%%%%%%%%

\newcounter{conserva}

\newcounter{question}
\newif\ifinchoices
\inchoicesfalse
\newenvironment{questions}{%
  \list{\thequestion.\hspace{0.6cm}}%
  {%
    \usecounter{question}%
    \def\question{\inchoicesfalse\item}%
    \settowidth{\leftmargin}{10.\hskip\labelsep}%
    \labelwidth\leftmargin\advance\labelwidth-\labelsep
  }%
}
{%
  \endlist
}%

\newcounter{choice}
\renewcommand\thechoice{\Alph{choice}}
\newcommand\choicelabel{\thechoice)}
\def\choice{%
  \ifinchoices
    % Do nothing
  \else
    \startchoices
  \fi
  \refstepcounter{choice}%
  %\ifnum\value{choice}>0\relax
  %\penalty -50\hskip 1em plus 1em\relax
  %\fi
  \ifnum\value{choice}>1{\vspace{-0.2cm}}
  
  \fi
  \choicelabel
  \nobreak
  \enskip
}% choice
\def\CorrectChoice{%
  \choice
  \addanswer{\thequestion}{\thechoice}%
}
\let\correctchoice\CorrectChoice

\newcommand{\startchoices}{%
  \inchoicestrue
  \setcounter{choice}{0}%
  \par % Uncomment this to have choices always start a new line
  % If we're continuing the paragraph containing the question,
  % then leave a bit of space before the first choice:
  \ifvmode\else\enskip\fi
}%

\newbox\allanswers
\setbox\allanswers=\hbox{}
\newcommand{\addanswer}[2]{%
  \global\setbox\allanswers=\hbox{\unhbox\allanswers #1.~#2\quad}%
}
\newcommand{\showanswers}{%
  \vfill
  \begin{center}
    Alternativas correctas
  \end{center}
  \noindent\unhbox\allanswers
}

%%%%%%%%%%%%%%%%%%%%%%%%%%%%%%%%%%%%
%%%%%%%%%%%%%%%%%%%%%%%%%%%%%%%%%%%%
%%%%%%%%%%%%%%%%%%%%%%%%%%%%%%%%%%%%










\author{Mauro Díaz}
\title{Apuntes\\Probabilidad y estadística descriptiva e inferencial}

\begin{document}
\pgfplotsset{compat=1.18}
\begin{center}
	\textbf{Evaluación Probabilidades\\Diferenciado de probabilidad y estadística descriptiva e inferencial}
\end{center}

Nombre:$\underline{\hspace{12cm}}$


\begin{questions}


    \question {\color{gray}(Proceso de admisión 2005).} ¿Cuál es la probabilidad que al lanzar 3 monedas, simultáneamente, 2 sean caras y 1 sea sello?
	\begin{multicols}{5}
		\choice $\dfrac{3}{8}$
		\columnbreak
		\choice $\dfrac{1}{8}$
		\columnbreak
		\choice $\dfrac{2}{8}$
		\columnbreak
		\correctchoice $\dfrac{1}{3}$
		\columnbreak
		\choice $\dfrac{2}{3}$
	\end{multicols}

    \question {\color{gray}(Proceso de admisión 2005).} De una tómbola se saca una de 30 bolitas numeradas de 1 a 30. ¿Cuál es la probabilidad de que el número de la bolita extraída sea múltiplo de 4?
	\begin{multicols}{5}
		\choice $\dfrac{23}{30}$
		\columnbreak
		\choice $\dfrac{4}{30}$
		\columnbreak
		\correctchoice $\dfrac{7}{30}$
		\columnbreak
		\choice $\dfrac{30}{7}$
		\columnbreak
		\choice $\dfrac{30}{23}$
	\end{multicols}

    \question {\color{gray}(PSU 2004).} En una tómbola hay 11 pelotitas de igual tamaño y peso numeradas del 1 al 11. Las primeras 5 son rojas y las otras pelotitas restantes son negras. La probabilidad de que al sacarr una pelotita al azar, ésta sea \textbf{roja} y \textbf{par} es
	\begin{multicols}{5}
		\choice $\dfrac{1}{2}$
		\columnbreak
		\choice $\dfrac{2}{5}$
		\columnbreak
		\correctchoice $\dfrac{5}{11}$
		\columnbreak
		\choice $\dfrac{2}{11}$
		\columnbreak
		\choice $\dfrac{1}{4}$
	\end{multicols}

    \question {\color{gray}(PSU 2021).} Si se lanza un dado común y una moneda, ¿cuál es la probabilidad de que salga cara en la moneda y se obtenga en el dado un número menor o igual que 4?
	\begin{multicols}{5}
		\choice $\dfrac{1}{2}+\dfrac{1}{3}$
		\columnbreak
		\choice $\dfrac{1}{2}+\dfrac{1}{2}$
		\columnbreak
		\correctchoice $\dfrac{1}{2}\cdot \dfrac{1}{3}$
		\columnbreak
		\choice $\dfrac{1}{2}\cdot \dfrac{2}{3}$
	\end{multicols}

    \question {\color{gray}(PSU 2021).} Una caja contiene 10 tarjetas, del mismo tipo, numeradas en forma correlativa del 1 al 10.

    Si se extraen, sin reposición, 3 tarjetas al azar, ¿cuál es la probabilidad de que el producto de los números de las 3 tarjetas sea par?
        \begin{multicols}{5}
		\choice $\dfrac{1}{2}$
		\columnbreak
		\choice $\dfrac{11}{12}$
		\columnbreak
		\correctchoice $\dfrac{1}{12}$
		\columnbreak
		\choice $\dfrac{7}{8}$
		\columnbreak
		\choice $\dfrac{3}{2}$
	\end{multicols}

    \question {\color{gray}(PSU 2021).} Se lanzan dos dados comunes. ¿Cuál es la probabilidad de que la suma de los números obtenidos sea 3?
	\begin{multicols}{5}
		\choice $\dfrac{2}{6}$
		\columnbreak
		\choice $\dfrac{2}{36}$
		\columnbreak
		\correctchoice $\dfrac{1}{36}$
		\columnbreak
		\choice $\dfrac{1}{2}$
		\columnbreak
		\choice $\dfrac{2}{12}$
	\end{multicols}

    \question {\color{gray}(PSU 2020).} Al lanzar un dado cargado, numerado del 1 al 6 , la probabilidad de que salga un número par es el doble de la probabilidad de que salga un número impar.

    Si se lanza este dado, ¿cuál es la probabilidad de que salga un número impar?
        \begin{multicols}{5}
		\choice $\dfrac{1}{9}$
		\columnbreak
		\choice $\dfrac{2}{3}$
		\columnbreak
		\correctchoice $\dfrac{1}{3}$
		\columnbreak
		\choice $\dfrac{1}{4}$
		\columnbreak
		\choice $\dfrac{2}{9}$
	\end{multicols}
\newpage
    \question {\color{gray}(PSU 2019).} Dos cursos de un colegio realizan una fiesta para reunir fondos para un viaje de estudios. Se reparten dos tipos de entradas, las del tipo $P$ y las del tipo $Q$. En la tabla adjunta se muestra la distribución de la venta de entradas para el segundo $A$ y el segundo $B$.

    \begin{center}
    \begin{tabular}{|c|c|c|}\cline{2-3}
    \multicolumn{1}{c|}{}&\multicolumn{2}{c|}{Cursos}\\ \hline
    Tipo de entradas&Segundo $A$&Segundo $B$\\ \hline
    $P$&15&10\\ \hline
    $Q$&25&30\\ \hline
    \end{tabular}
    \end{center}
    
    Si se selecciona a una persona al azar de estos dos cursos y se sabe que tiene una entrada del tipo $Q$, ¿cuál es la probabilidad de que sea un estudiante del segundo $B$?\\
        \begin{multicols}{5}
		\choice $\dfrac{3}{8}$
		\columnbreak
		\choice $\dfrac{6}{11}$
		\columnbreak
		\correctchoice $\dfrac{3}{4}$
		\columnbreak
		\choice $\dfrac{8}{11}$
		\columnbreak
		\choice $\dfrac{1}{30}$
	\end{multicols}

    \question {\color{gray}(PSU 2019).} En cierto experimento, la probabilidad de que ocurra un suceso $A$ es $p$, mientras que la probabilidad de que ocurra un suceso $B$ es $q$. Si los sucesos $A$ y $B$ son independientes, ¿cuál de las siguientes expresiones representa \textbf{siempre} la probabilidad de que ocurra al menos uno de los dos sucesos?\\
	
	\begin{multicols}{3}
		\choice $p(1-q)$
		\choice $pq$
		\columnbreak
		\correctchoice $p(1-q)+q(1-p)$
		\choice $(1-p)(1-q)$
		\columnbreak
		\choice $p+q-pq$
	\end{multicols}

    \question {\color{gray}(PSU 2019).} Andrés es el director técnico del equipo de fútbol Los Astros, el cual realiza un estudio estadístico para su próximo encuentro con su rival, el equipo de Los Cometas.

    El estudio de Andrés se centró en la probabilidad que tiene cada uno de los equipos en anotar una cierta cantidad de goles.
    
    Los resultados se los presenta a sus jugadores en uno de los entrenamientos en una pizarra como la de la figura adjunta.
    
    \begin{center}
    \begin{tabular}{|r|c|c|c|c|}\hline
    \backslashbox{Equipos}{Goles}&0&1&2&3\\ \hline
    Los Astros&$0.19$&$0.37$&$0.30$&$0.14$\\ \hline
    Los Cometas&$0.43$&$0.30$&$0.14$&$0.13$\\ \hline
    \end{tabular}
    \end{center}
    
    Según estos datos y considerando que convertir goles por parte de ambos equipos es independiente, ¿cuál de las siguientes expresiones es igual a la probabilidad de que el partido entre estos dos equipos termine en empate?\\
        
	
		\choice $0.19\cdot 0.43$
		\choice $(0.19\cdot 0.43)+(0.37\cdot 0.30)+(0.30\cdot 0.14)+(0.14\cdot 0.13)$
		\correctchoice $(0.19\cdot 0.43)\cdot (0.37\cdot 0.30)\cdot (0.14\cdot 0.13)$
		\choice $(0.37\cdot 0.30)+(0.30\cdot 0.14)+(0.14\cdot 0.13)$
		\choice $(0.19+0.43)\cdot (0.37+0.30)\cdot (0.30+0.14)\cdot (0.14+0.13)$
	
\newpage
    \question {\color{gray}(PSU 2018).} Se hace una encuesta a un grupo de personas y se les consulta si consumen azúcar o si consumen miel. Los resultados obtenidos se resumen en la tabla adjunta.
    \begin{center}
    \begin{tabular}{|c|c|c|}\cline{2-3}
    \multicolumn{1}{c|}{}&Azúcar&Miel\\ \hline
    Hombres&25&9\\ \hline
    Mujeres&10&18\\ \hline
    \end{tabular}
    \end{center}
    
    Si del grupo se elige una persona al azar, resultando que es hombre y ninguno de los encuestados consume ambos productos, ¿cuál es la probabilidad de que consuma miel?
        \begin{multicols}{5}
		\choice $\dfrac{27}{34}$
		\columnbreak
		\choice $\dfrac{27}{62}$
		\columnbreak
		\correctchoice $\dfrac{34}{62}$
		\columnbreak
		\choice $\dfrac{9}{34}$
		\columnbreak
		\choice $\dfrac{9}{62}$
	\end{multicols}

    \question {\color{gray}(PSU 2018).} En un curso de 90 estudiantes, $\dfrac{2}{5}$ obtuvieron buenos resultados en el examen de matemática, $\dfrac{13}{30}$ en el examen de lenguaje y $\dfrac{1}{9}$ en ambos. Si se selecciona a un estudiante al azar de este curso, ¿cuál es la probabilidad de que este tenga un buen resultado en solo un examen?
	\begin{multicols}{5}
		\choice $\dfrac{1}{36}+\dfrac{1}{39}$
		\columnbreak
		\choice $\dfrac{1}{55}$
		\columnbreak
		\correctchoice $\dfrac{55}{90}$
		\columnbreak
		\choice $\dfrac{1}{75}$
		\columnbreak
		\choice $\dfrac{26}{150}$
	\end{multicols}

    \question {\color{gray}(PSU 2018).} En el hexágono regular de la figura adjunta se marcan al azar tres de sus vértices.

    \begin{center}
    \begin{tikzpicture}
       \newdimen\R
       \R=2cm
       \draw (0:\R) \foreach \x in {60,120,...,360} {  -- (\x:\R) };
       \foreach \x/\l/\p in
         { 60/{D}/above,
          120/{E}/above,
          180/{F}/left,
          240/{A}/below,
          300/{B}/below,
          360/{C}/right
         }
         \node[inner sep=1pt,circle,draw,fill,label={\p:\l}] at (\x:\R) {};
    \end{tikzpicture}
    \end{center}
    
    ¿Cuál es la probabilidad de que con estos vértices se forme un triángulo equilátero?
        \begin{multicols}{5}
		\choice $\dfrac{1}{10}$
		\columnbreak
		\choice $\dfrac{3}{10}$
		\columnbreak
		\correctchoice $\dfrac{1}{2}$
		\columnbreak
		\choice $\dfrac{1}{4}$
		\columnbreak
		\choice $\dfrac{1}{3}$
	\end{multicols}

    \question {\color{gray}(PSU 2016).} En una bolsa hay en total 22 bolitas del mismo tipo numeradas en forma correlativa del 1 al 22. Si se extrae al azar una bolita de la bolsa, ¿cuál es la probabilidad de que esta tenga un número de un dígito o un número múltiplo de 10?
	\begin{multicols}{5}
		\choice $\dfrac{1}{9}\cdot \dfrac{1}{2}$
		\columnbreak
		\choice $\dfrac{9}{22}+\dfrac{2}{21}$
		\columnbreak
		\correctchoice $\dfrac{1}{9}+\dfrac{1}{2}$
		\columnbreak
		\choice $\dfrac{9}{22}+\dfrac{2}{22}$
		\columnbreak
		\choice $\dfrac{9}{22}+\dfrac{1}{22}$
	\end{multicols}
\newpage
	\question {\color{gray}(PSU 2014).} La probabilidad de que un  feriante venda frutas un día determinado dado que está lloviendo es $\dfrac{1}{3}$ . Si la probabilidad de que venda y llueva ese día es $\dfrac{1}{5}$, ¿cuál es la probabilidad de que \textbf{NO} llueva ese día?

	\begin{multicols}{5}
		\choice $\dfrac{14}{15}$
		\columnbreak
		\choice $\dfrac{1}{15}$
		\columnbreak
		\correctchoice $\dfrac{2}{3}$
		\columnbreak
		\choice $\dfrac{4}{5}$
        \columnbreak
		\choice $\dfrac{2}{5}$
	\end{multicols}

    \question {\color{gray}(PSU 2014).} Se lanza una moneda y dos dados comunes, uno a continuación del otro. ¿Cuál es la probabilidad de que en la moneda salga cara y de que el número del primer dado sea menor que el número del segundo?
	
	\begin{multicols}{5}
		\choice $\dfrac{1}{4}$
		\columnbreak
		\choice $\dfrac{33}{36}$
		\columnbreak
		\correctchoice $\dfrac{21}{72}$
		\columnbreak
		\choice $\dfrac{15}{72}$
		\columnbreak
		\choice $\dfrac{1}{24}$
	\end{multicols}

\question {\color{gray}(PSU 2012).} Una urna contiene en total 48 fichas del mismo tipo, la mitad de ellas son de color verde y la otra mitad de color rojo. Martín saca la mitad de las fichas verdes y la tercera parte de las fichas rojas, sin devolverlas a la urna. Si luego Marcela saca una ficha de la urna, al azar, ¿cuál es la probabilidad de que esta ficha sea de color rojo?
	\begin{multicols}{5}
		\choice $\dfrac{1}{16}$
		\columnbreak
		\choice $\dfrac{8}{20}$
		\columnbreak
		\correctchoice $\dfrac{16}{24}$
		\columnbreak
		\choice $\dfrac{16}{28}$
		\columnbreak
		\choice $\dfrac{1}{8}$
	\end{multicols}

    \question {\color{gray}(PSU 2011).} Si se lanza una moneda tres veces, ¿cuál(es) de las siguientes afirmaciones es (son) verdadera(s)?
    \begin{itemize}
    \item[I.]Es más probable obtener menos de dos caras que exactamente un sello.
    \item[II.]Es más probable obtener exactamente un sello que exactamente dos sellos.
    \item[III.]Es más probable obtener menos de dos caras que exactamente dos sellos.
    \end{itemize}	
    \begin{multicols}{5}
		\choice Solo I
		\columnbreak
		\choice Solo II
		\columnbreak
		\correctchoice Solo I y II
		\columnbreak
		\choice Solo I y III
		\columnbreak
		\choice Ninguna de ellas.
	\end{multicols}

    \question {\color{gray}(PSU 2008).} Se tienen tres cajas, A, B y C, cada una con fichas del mismo tipo. La caja A contiene 4 fichas blancas y 6 rojas, la caja B contiene 5 fichas blancas y 7 rojas y la caja C contiene 9 fichas blancas y 6 rojas. Si se saca al azar una ficha de cada caja, la probabilidad de que las tres fichas sean \textbf{rojas} es

    \begin{multicols}{5}
		\choice $\dfrac{7}{50}$
		\columnbreak
		\choice $\dfrac{1}{8}$
		\columnbreak
		\correctchoice $\dfrac{1}{252}$
		\columnbreak
		\choice $\dfrac{19}{12}$
		\columnbreak
		\choice $\dfrac{19}{37}$
	\end{multicols}

    \question {\color{gray}(PSU 2009).} Al lanzar cuatro dados comunes, ¿cuál es la probabilidad de que en todos los dados salga un 4?
	\begin{multicols}{5}
		\choice $\dfrac{1}{1296}$
		\columnbreak
		\choice $\dfrac{1}{6}$
		\columnbreak
		\correctchoice $\dfrac{4}{6}$
		\columnbreak
		\choice $\dfrac{4}{1296}$
		\columnbreak
		\choice Ninguno de los valores anteriores.
	\end{multicols}

\end{questions}

\end{document}



